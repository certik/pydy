\documentclass[letterpaper,11pt]{article}
\usepackage[round]{natbib}
\usepackage[margin=1in,centering]{geometry}
\usepackage{fancyhdr}
\usepackage{amsmath}
\usepackage{amssymb}
\usepackage{graphicx}
\usepackage[pdftex]{hyperref}
\hypersetup{
    pdftitle={Analysis of a Gyrostat model},
    pdfauthor={Dale Lukas Peterson},
    pdfsubject={Subject},
    pdfkeywords={Keywords}}

\newcommand{\bs}[1]{ \boldsymbol{ #1 } }
\newcommand{\ud}{\mathrm{d}}
\pagestyle{fancy}
\fancyhead[L]{Gyrostat equations of motion, Dale L. Peterson}
\fancyhead[R]{\today}  % page number on the right
\fancyfoot[L,R]{}  %  No footer on left, center or right, on even or odd pages
\fancyfoot[C]{\thepage}
\begin{document}
\begin{abstract}
  Equations of motion for a gyrostat comprised of two rigid bodies (a carrier and
  a rotor) are derived in two different ways.  In the first approach, each
  rigid body is treated separately as each having their own mass and
  inertia.  In the second approach, the symmetry of the rotor is used to
  combine the inertias in such a way that simplifies the derivation of the
  equations of motion.  It is shown that the two approaches are equivalent.
  For generality, the axis of rotation of the rotor is not aligned with a
  principal axis of the carrier.
\end{abstract}

\section*{Model Description}

Consider a rigid body $A$, of mass $m_a$, along with a dextral set of unit
vectors $\bs{a}_1$, $\bs{a}_2$, $\bs{a}_3$, fixed to $A$, with $\bs{a}_3
\triangleq \bs{a}_1 \times \bs{a}_2$.  Let $A^*$ denote the center of mass of
$A$.  Let the inertia matrix of $A$ relative to its mass center $A^*$ for
$\bs{a}_1, \bs{a}_2, \bs{a}_3$ be:

\begin{align*}
  I^{A/A^*} & =
    \left[ \begin{array}{c c c}
      I_{11} & I_{12} & I_{13} \\
      I_{12} & I_{22} & I_{23} \\
      I_{13} & I_{23} & I_{33}
    \end{array} \right]
\end{align*}

Consider a second rigid body, $B$, of mass $m_b$, along with a dextral set of unit
vectors $\bs{b}_1$, $\bs{b}_2$, $\bs{b}_3$, fixed to $B$, with $\bs{b}_3
\triangleq \bs{b}_1 \times \bs{b}_2$.  Let $B^*$ denote the center of mass of
$B$.  Let the inertia matrix of $B$ relative to its mass center  $B^*$ for
$\bs{b}_1, \bs{b}_2, \bs{b}_3$ be:

\begin{align*}
  I^{B/B^*} & =
    \left[ \begin{array}{c c c}
      I & 0 & 0 \\
      0 & J & 0 \\
      0 & 0 & K
    \end{array} \right]
\end{align*}

The mass center $B^*$ is located relative to $A^*$ by:
\begin{align*}
  r^{B^* / A^*} & = l_1 \bs{a}_1 + l_2 \bs{a}_2 + l_3 \bs{a}_3
\end{align*}

Where $l_1, l_2, l_3$ are real constants.  Denote the center of mass of the
system as $AB^*$. The position of $AB^*$ relative to the mass centers of $A$
and $B$, is:
\begin{align*}
  r^{AB^* / A^*} & = (l_1 m_b\bs{a}_1 + l_2 m_b\bs{a}_2 + l_3
  m_b\bs{a}_3)/(m_a+m_b)\\
  r^{AB^* / B^*} & =  -(l_1 m_a\bs{a}_1 + l_2 m_a\bs{a}_2 + l_3
  m_a\bs{a}_3)/(m_a+m_b)
\end{align*}

$B$ is oriented relative to $A$ by means of a revolute joint whose axis is
aligned with the $\bs{a}_2$ and passes through $B^*$.  This implies that
$\bs{a}_2 = \bs{b}_2$.

A couple of torque $T$ is applied to by $A$ on $B$ about the axis of rotation
$\bs{a}_2$.

\section*{Equations of motion}
The system is completely described by seven generalized coordinates: three which
locate the system mass center $AB^*$, three which orient $A$ (alternatively
$B$) in inertial space, and one which orients $B$ relative to $A$.  Let $\omega^A$ and
$\omega^B$ denote the angular velocity of $A$ and $B$ relative to the inertial
frame, respectively.  Let $v^{AB^*}$ denote the velocity of the center of mass
of the system relative to the inertial frame.  Define the following generalized
speeds:
\begin{align}
  u_i & \triangleq \omega^A \cdot \bs{a}_i  \qquad  (i = 1,2,3) \label{u_defs1} \\
  u_4 & \triangleq \omega^B \cdot \bs{a}_2 \label{u_defs2} \\
  u_i & \triangleq v^{AB^*} \cdot \bs{a}_{i-4} \qquad  (i = 5,6,7) \label{u_defs3}
\end{align}

Consider orienting $A$ relative to the inertial frame $N$ using $(q_1, q_2,
q_3)$ Euler 3-1-2 angles, and orient $B$ relative to $A$ by a right handed
rotation by an angle $q_4$ about the $\bs{a}_2 = \bs{b}_2$ axis. Locate the
system mass center relative to the inertial frame by the vector $r^{AB^*/N*} =
q_5\bs{n}_1 + q_6\bs{n}_2 + q_7\bs{n}_3$.  These definitions, along with
(\ref{u_defs1}, \ref{u_defs2}, \ref{u_defs3}) yield the following kinematic
differential equations:

\begin{align*}
  \dot{q}_1 &=  -s_3u_1/c_2 + c_3u_3/c_2\\
  \dot{q}_2 &=  c_3u_1 + s_3u_3 \\
  \dot{q}_3 &=  t_2s_3u_1 + u_2 - t_2c_3u_3 \\
  \dot{q}_4 &= -u_2 +  u_4 \\
  \dot{q}_5 &= (c_1c_3 - s_1s_2s_3)u_5 - c_2s_1u_6 + (c_1s_3 + c_3s_1s_2)u_7\\
  \dot{q}_6 &= (c_3s_1 + c_1s_2s_3)u_5 + c_1c_2u_6 + (s_1s_3 - c_1c_3s_2)u_7\\
  \dot{q}_7 &= -c_2s_3u_5 + s_2u_6 + c_2c_3u_7
\end{align*}

While these equations are necessary for kinematics, they aren't too interesting
because all the coordinates are ignorable with respect to the dynamic
differential equations (no coordinates appear in the dynamic equations of
motion).  Other parameterizations of orientation are certainly possible, and in
fact desirable to avoid the singularities of the Euler 3-1-2 angles when $q_2 =
\pm \pi/2$.

\end{document}
